\documentclass[fr]{../../../eplsummary}

\usepackage{../../../eplcode}
\usepackage{syntax}
\usepackage{calc}
\usepackage{svg}
\usepackage{float}

%code Oz
\usepackage{color}
\lstset{language={Oz},morekeywords={for,do,lazy}}

% To have a proper indent for EBNF rules
\newlength{\myl}

\newenvironment{indentgrammar}[1]
    {\setlength{\myl}{\widthof{#1}+2em}
    \grammarindent\the\myl
    \begin{grammar}}
    {\end{grammar}}
% End EBNF indent

\hypertitle{Computer language concepts}{6}{INGI}{1131}
{Alexandre Gobeaux\and Gilles Peiffer\and Liliya Semerikova}
{Peter Van Roy}

\newpage % separate toc

\section{Rappels}

\subsection{Vue d'ensemble des paradigmes}
\begin{figure}[htbp]
  \centering
  \includesvg[width=0.83\textwidth]{img/ParadigmsOverview.svg}
  \caption{Vue d'ensemble des paradigmes.}
\end{figure}

\begin{figure}[H]
\centering
\begin{subfigure}[t]{0.47\textwidth}
\centering
\includesvg[width=\textwidth]{img/PortMessages.svg}
\caption{Figure représentative d'un \emph{port} (canal de communication).}
\end{subfigure}\hfill
\begin{subfigure}[t]{0.47\textwidth}
\centering
\includesvg[width=\textwidth]{img/multiAgent.svg}
\caption{Programmation multi-agents}
\end{subfigure}
\caption{Explication des ports et de la programmation multi-agents.}
\end{figure}

\subsection{Programmation fonctionnelle}
En programmation fonctionnelle:
\begin{itemize}
	\item Il n'y a pas de \emph{variables muables};
	les variables sont à \emph{affectation unique}.
	\item Un programme est une fonction ($z = f(x,y)$).
	\item Une \emph{fonction} retourne un \emph{résultat},
	alors qu'une \emph{procédure} n'effectue qu'une \emph{action},
	et ne renvoie \emph{pas de résultat}.
	\item Un programme qui fonctionne fonctionnera toujours.
\end{itemize}

\subsection{Structures de données}
Une \emph{structure de données} regroupe des données.
Un exemple de structure de données est la \emph{liste}.
La définition d'une liste
avec les règles EBNF\footnote{\emph{Extended Backus--Naur Form}.}
est la suivante:
\begin{indentgrammar}{<list>}
<list> ::= nil
\alt <e> '|' <list>
\end{indentgrammar}
On note que \synt{<list>} est une expression non terminale,
et est définie en fonction d'elle-même (récursivement).
Avec le sucre syntaxique, \lstinline!1|nil! peut s'écrire aussi \lstinline![1]!.

\begin{figure}[H]
\centering
\includesvg[scale=0.5]{img/567Tree.svg}
\caption{Exemple de représentation d'une liste.}
\label{fig:tree}
\end{figure}
La \figuref{tree} représente la liste \lstinline!L = 5|6|7|nil!.
On a donc comme premier élément de la liste (\emph{head}) \lstinline|L.1 = 5|
et comme reste de la liste (\emph{tail}) \lstinline!L.2 = 6|7|nil!.
Une autre façon de représenter une liste
est en utilisant la notation standard des \emph{enregistrements (records)}:
\lstinline!L = '|'(5 '|'(6 '|'(7 nil)))!.
Avec le sucre syntaxique,
cette liste deviendrait simplement \lstinline|L = [5 6 7]|.

\subsection{Fonctions, conditions et récursivité}
\subsubsection{Fonction}
L'exemple ci-dessous est une fonction en \oz;
il s'agit de la fonction \lstinline!F! avec comme argument \lstinline!X!.
On pourrait remplacer \lstinline!<expression>! par \lstinline!X.1!
si \lstinline!X! est une liste afin de renvoyer
le premier élément de \lstinline!X! par exemple.
\begin{lstlisting}
fun {F X}
	<expression>
end
\end{lstlisting}

\subsubsection{Condition}
Ci-dessous, on observe la syntaxe d'une \emph{condition} en \oz.
\begin{lstlisting}
if <expr> then <expr>
else <expr>
end
\end{lstlisting}

\subsubsection{Récursivité}
En assemblant les deux concepts précédents,
on peut définir une fonction et définir la notion de \emph{récursivité}.
Une fonction est \emph{récursive} si elle s'appelle elle-même.
Ci-dessous, on observe la fonction \lstinline|Sum|
qui fait la somme des éléments de la liste \lstinline|X| récursivement
(on note l'appel à \lstinline|Sum| sur la tail de la liste).
\begin{lstlisting}
fun {Sum X}
	if X == nil then 0
	else X.1 + {Sum X.2}
	end
end
\end{lstlisting}
On remarque la correspondance entre le concept de règle de grammaire récursive
et celui de fonction récursive.

On peut également parler de \emph{récursivité terminale}:
une fonction est \emph{récursive terminale (tail recursive)}
si l'appel récursif est la dernière opération.
On peut donc réécrire le programme \lstinline|Sum|
pour qu'il soit récursif terminal en utilisant un \emph{accumulateur}
(identificateur qui va accumuler le résultat au fur et à mesure de l'exécution):
\begin{lstlisting}
fun {SumTailRecursive X Acc}
	if X == nil then Acc
	else {SumTailRecursive X.2 X.1+Acc}
	end
end

fun {Sum X}
	{SumTailRecursive X 0}
end
\end{lstlisting}
Ce programme sera plus efficace, car il ne doit pas stocker en mémoire
tous les résultats intermédiaires grâce à l'accumulateur.

\subsection{Représentation en mémoire}
\begin{figure}[H]
	\centering
	\includesvg[scale = 0.5]{img/ProgEnvMem.svg}
	\caption{Représentation en mémoire d'un programme.}
	\label{fig:ProgEnvMem}
\end{figure}
Sur la \figuref{ProgEnvMem}, on peut voir que l'\emph{environnement}
fait une sorte de \emph{mapping} entre les \emph{identificateurs}
et les \emph{variables} stockées dans la \emph{mémoire}.
Lorsque \lstinline|X| est déclaré pour la première fois,
l'environnement le mappe sur la variable $x$ en mémoire,
initialisée à $10000$ par le code de l'exemple.
Lorsqu'ensuite, on veut exécuter \lstinline|X = 999|,
l'identificateur \lstinline|X| est déjà lié à la variable $x$ qui vaut $10000$.
Une erreur se produit donc, car $999 \ne 10000$.
Si, à la place, on déclare à nouveau \lstinline|X| avec \lstinline|declare|,
cette fois-ci, on crée une nouvelle variable en mémoire, $x'$,
qui sera initialisée à $999$.
La variable $x$ est devenue inaccessible.

\subsection{Réflexions sur la programmation fonctionnelle}
La programmation fonctionnelle est à \emph{affectation unique},
ce qui est une grosse restriction.
Afin d'écrire des programmes \emph{efficaces},
il faut utiliser des fonctions récursives terminales,
car celles-ci ne font pas grandir la \emph{pile} de l'ordinateur.
La programmation fonctionnelle est aussi efficace
que la programmation avec variables muables.
Nous n'avons pas besoin de celles-ci pour faire un programme efficace.

\end{document}
